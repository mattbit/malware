\documentclass{essay}

\usepackage{lipsum}

\usepackage{amsmath,amssymb}
\usepackage{csquotes}
\usepackage[backend=bibtex,style=numeric]{biblatex}
\addbibresource{bibliography.bib}
\renewcommand*{\bibfont}{\small\sffamily}

\title{Web malware detection with Markov random fields\\and belief propagation}
\author{\textbf{Matteo Dora} \enspace\textbullet\enspace matteo.dora@ens.fr}
\date{\today}

\begin{document}
\maketitle

\color{dark}
\sffamily

{\bfseries\sffamily\noindent As modern society has become increasingly dependent on digital techologies and communication networks, malicious software emerges as a major threat. Nefariously famous computer worms such as Code Red or Storm have caused economic losses estimated in the billions of USD \cite{karyotis_malware_2016}@todo(add references)and sophisticated state-sponsored attacks have leaked business secrets \cite{wp_sony_hack} or damaged military targets \cite{stuxnet}. Due to their increasing popularity, websites have become one of the most frequent attack vectors. Extensive research has been conducted to detect malicious web domains based on their behaviour \cite{zhauniarovich_survey_2018}. Here I follow a different approach, by analysing the topology of the interaction network between users and web domains, and I review a simple Markov random field (\textsc{MRF}) that allows to detect malicious domains and hosts, evaluating its predictive accuracy via numerical simulations.}

\section{Introduction}

Malicious software has been around since the days of yore, but it is only in the last two decades that it became a highly profitable business. The massive amount of sensible information that is nowadays available on digital devices make for an attractive loot to cybercriminals, and the internet provides a very effective medium for propagation. With the profitability growth, more and more resources have been dedicated to malware development, leading to increasingly advanced tools. Modern malware can be roughly divided into two main categories. On one hand, criminal organisations and intelligence agencies sponsored the creation of highly sophisticated attack tools. Examples of this kind of malware are Stuxnet \cite{stuxnet}, a worm targeting Iran's nuclear infrastructure, and Zeus, a criminal botnet dismantled in 200x with more than 100 arrests worldwide @todo(add reference). On the other hand, older versions of advanced malware and lesser evoluted tools began circulating in a thriving black market, allowing small criminal groups and individuals to approach the business. These kind of malware is usually directed at novice users and diffuses through email, websites and social networks. In the following, I will focus on this latter category of less sophisticated but widely spread malware, altough some of the methods described may be used in more advanced contexts.

In the recent years, web based malware has become the most popular threat \cite{liu_web_2017} @todo(add references). This breed of malware is usually injected into target devices by malicious web pages. It can spread by inducing users into following malicious hyperlinks, typically distributed via social networks or email. Other common attack vectors are legitimate websites compromised by means of malicious techniques such as cross site scripting (\textsc{xss}) or \textsc{sql} injection and configured to diffuse malware. Malicious web domains have usually a relatively short life, as once identified they can immediately taken down by authorities or blocked by antivirus software. Nonetheless, cybercriminals are continuously registering new malicious domains (or compromising existing ones), making it an infinite hide-and-seek game. How can web threats be prevented? Is it possible to identify malicious domains at early stage, before most of the damage is done?

Here, I will present a simple model for malware spreading based on Markov random fields and review an inference method, previously explored by Manadhata et al. \cite{manadhata_detecting}, that allows to identify malicious web domains based on the traffic exchanged in a local network.

\section{Modelling malicious web domains}

Many approaches modelling malware epidemics have been proposed based on classical compartmental models such as \textsc{sis} and \textsc{sir} \cite{liu_web_2017,gil2014genetic,karyotis_malware_2016}. Although these deterministic models are valid and have proven successful (for example in the case of the Code Red worm \cite{zou_code}), they can only describe the global characteristics of the infection diffusion. Stochastic epidemic models would overcome most of the limitations of deterministic ones, but analytical analysis (e.g. using queuing theory) is often intractable. Moreover, both categories are extremely sensible to variations in the topology of the system, which can affect the system ergodicity, making these models an inconvenient choice when dealing with complex networks.

Following the framework proposed by Karyotis \cite{karyotis_malware_2016,karyotis2017markov}, here I will focus on a simpler model constituted by a Markov random field (\textsc{mrf}). A \textsc{mrf} is network of random variables interacting only with their neighbours. Despite their simplicity, \textsc{mrf}s are good candidates to describe a malware propagation. The Markov spatial property is a sensible assumption, as malware can only spread by direct contact. Also, they are extremely robust to topological variations, as each variable, given its neighbours, is conditionally independent of any other node. This latter characteristic make them very useful to model processes on different kind of networks.


To describe malware diffusion we consider the \textsc{SIS} paradigm and build a \textsc{mrf} where nodes can assume two states: susceptible ($x = +1$) or infected ($x = -1$). Nodes are organised on a network where links represent data exchange between the nodes. Intuitively, susceptible nodes communicating with infected neighbours will turn infected with high probability. This kind of interaction can be described by an Ising Hamiltonian, such that:
\begin{align}
& \mathbb{P}\left(X\right) = \frac{1}{Z}e^{-\beta H\left(X\right)}\\
& H\left(X\right) = - J \sum_{\langle i, j\rangle} x_i x_j - \sum_i h_i x_i
\end{align}
where $X$ denotes a system configuration $(x_0, x_1, x_2, \dots)$, $J$ is the interaction potential and $h_i$ is the local field of node $i$. In this setting, all the nodes are in principle free to switch between the two states. Yet, we want to represent attackers as infected nodes that never change state. We can do that by setting their local field to infinite (in this case, $h_{attacker} = - \infty$). The same technique in the case we want to fix the state of susceptible nodes.

\subsection{Network topology}

\begin{figure}
\end{figure}

Let us suppose that, given a local network, we can intercept every request coming from local hosts and directed to domains in the outer web (e.g. through a \textsc{http} proxy, as described in \cite{manadhata_detecting}). Using this knowledge, we can build a bipartite graph linking local hosts to the remote domains they interact with. In a real network, these links will not be uniformly distributed, but will depend on the popularity of the domains. It has been shown that in many cases where popularity influences the connectivity (preferential attachment), one obtains a scale-free graph. Examples of this kind of graph are social networks, and the network of web pages. Which are the characteristics of the interaction network between users (local hosts) and web domains? To answer this question, I analysed a dataset of traffic recorded at the Czech Technical University




\section{Methods}
\vfill
\nocite{*}
\printbibliography

\end{document}
